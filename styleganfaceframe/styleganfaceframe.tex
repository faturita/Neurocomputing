\documentclass[review]{elsarticle}

\usepackage{lineno,hyperref}
\modulolinenumbers[5]

\journal{Journal of \LaTeX\ Templates}

%%%%%%%%%%%%%%%%%%%%%%%
%% Elsevier bibliography styles
%%%%%%%%%%%%%%%%%%%%%%%
%% To change the style, put a % in front of the second line of the current style and
%% remove the % from the second line of the style you would like to use.
%%%%%%%%%%%%%%%%%%%%%%%

%% Numbered
%\bibliographystyle{model1-num-names}

%% Numbered without titles
%\bibliographystyle{model1a-num-names}

%% Harvard
%\bibliographystyle{model2-names.bst}\biboptions{authoryear}

%% Vancouver numbered
%\usepackage{numcompress}\bibliographystyle{model3-num-names}

%% Vancouver name/year
%\usepackage{numcompress}\bibliographystyle{model4-names}\biboptions{authoryear}

%% APA style
%\bibliographystyle{model5-names}\biboptions{authoryear}

%% AMA style
%\usepackage{numcompress}\bibliographystyle{model6-num-names}

%% `Elsevier LaTeX' style
\bibliographystyle{elsarticle-num}
%%%%%%%%%%%%%%%%%%%%%%%

\usepackage{amsmath}
\usepackage{amsfonts}
\usepackage{booktabs}

\begin{document}

\begin{frontmatter}

\title{Controlling Face's Frame generation in StyleGAN's latent space operations}

\author[ITBA]{Roca Agustín}
\ead{aroca@itba.edu.ar}

\author[ITBA]{Britos Nicolás Ignacio}
\ead{nbritos@itba.edu.ar}

\affiliation[ITBA]{organization={Instituto Tecnologico de Buenos Aires},
city={Ciudad Autonoma de Buenos Aires},
country={Argentina}}


\begin{abstract}
Some studies suggest that the face frame (the hairline, chin and ears), are more important 
for face recognition than the inner facial features (eyes, mouth, nose), especially in 
unfamiliar faces~\cite{want2003}. % TODO: incluir mas citas
In this paper, we define a way to measure the difference of face frame
between two face images, and use that to implement a better preservation of the face frame 
in the main StyleGAN~\cite{stylegan2} latent space operations (moving through latent directions and 
projection). With the tested images, an additional preservation between 20\% and 50\% was % TODO: Capaz podriamos implementar para direcciones también
obtained for the projection. This can be used to make the faces resulting from these operations
similar to their input. 
\end{abstract}

\begin{keyword}
StyleGAN, Generative adversarial network, Latent space, Image generation, Image Segmentation, Image Processing, Face frame
\end{keyword}

\end{frontmatter}

\linenumbers

\section{Introduction}

% TODO: Expand on memories studies results, neural networks to solve this kind of problems
% StyleGAN2 impact especially, contributions of this work, applications where this can be useful
% importance of exploring the latent space

\section{Materials and Methods}

% TODO: Expand on how to measure face frame difference, including the segmentation of the face.
% Expand on the correction of face frame in the projection operation, and maybe in the directions too.

\section{Results} \label{Results}
\label{section:results}

% TODO: Show graphs of face frame difference moving through directions, and when projecting with vs without correction


\section{Discussion}
\label{discussion}

% TODO: Interpretation of results, limitations

\section{Conclusions}

% TODO: Closure, summary of main takeaways of the paper

\bibliography{styleganfaceframe}

\end{document}